\section{Realisierungsplanung}
% Sind alle wichtigen Ziele/Meilensteine definiert? 
Einer der wichtigsten Meilensteine f"ur das Unternehmen ist die Herstellung
eines Prototyps der Maschine durch einen Modellbauer.
Zeitgleich kann bereits damit begonnen werden, die Firmware der Maschine zu
entwickeln, da im vorhinein bekannt ist, welche Komponenten verbaut werden
sollen.
Zuvor sollten jedoch bereits die Schnittstellen zwischen der Firmware und den
Steuerungsapps geplant und fest definiert werden, um keine Zeit durch
nachtr"agliche "Anderungen zu verlieren.
Sobald die Schnittstellen zwischen der Firmware und den Apps zur
Verf"ugung stehen, kann hier auch parallel mit der Entwicklung
begonnen werden.

Die Entwicklung von Software ist allgemein sehr aufw"andig und damit teuer. 
W"ahrend der Entwicklungsphase werden viele Resourcen verbraucht aber keine
Gewinne erwirtschaftet. 
Hier sind wir gezwungen, ein gro"ses Risiko einzugehen. 
Wir werden zun"achst einen Gro"steil der finanziellen Mittel und viele
Arbeitsstunden in die Entwicklung der eigentlichen Software in Form der Apps zur
Steuerung der Kaffeevollautomaten und der Firmware f"ur die Ger"ate investieren.

W"ahrend der Entwicklung ist es zwar wichtig, die Kosten in einem gewissen Budget zu halten
und auch die Firmware samt Steuerungsapps in angemessener Zeit fertigzustellen,
jedoch darf hier nicht an der falschen Stelle gespart werden oder aus
finanziellen Gr"unden auf eine Testfreigabe gedr"angt werden.
Unfertige Soft- und Firmware, welche Komponenten im Umgang mit hei"sen
Fl"ussigkeiten steuert, kann zu eventuellen Sch"aden und Verletzungen f"uhren.
Daher sollten zuerst kritische Funktionen durch die Entwickler getestet werden,
bevor au"senstehende Tester mit dem Produkt in Ber"uhrung kommen.

Die Tests der Soft- und Firmware in Kombination mit den Kaffeevollautomaten
werden "uber einen Zeitraum von zwei bis drei Wochen von Verwandten
durchgef"uhrt.
Das bietet eine realistische Testumgebung, da die Verwandten nicht
alle "uber gro"ses technisches Wissen verf"ugen. 
Daher k"onnen wir nach diesen Tests sicher sein, dass auch technisch unbedarfte 
Personen mit unserem System leicht zurechtkommen. 
Dies ist auch ein erkl"artes Ziel unseres Unternehmens: Wir wollen dem Kunden
Arbeit abnehmen sowie eine Zeitersparnis bieten und einen einfach zu bedienenden
und voll individualisierbaren Kaffeevollautomaten erstellen.

Sind die Systeme nach der Testphase stabil, kann mit der Produktion des
eigentlichen Kaffeevollautomaten und den mit NFC-Chips ausgestatteten Tassen
begonnen werden.
Dazu k"onnen dann auch studentische Aushilfskr"afte diese Arbeit f"ur
gr"o"sere Auftr"age "ubernehmen, um die Gesch"aftsf"uhrung zu entlasten.

Um nicht nur Erl"os durch den Verkauf der Kaffeevollautomaten zu erlangen,
wollen wir zudem neue und mit NFC-Chips best"uckte Tassen verkaufen.
Diese sollen eventuell besch"adigte Tassen ersetzen.
Es sollen auch NFC-Chips einzeln verkauft werden, damit der Kunde den
Automatismus unserer Kaffeevollautomaten mit seinen Lieblingstassen nutzen kann.

% Stehen die Meilensteine im richtigen zeitlichen Ablauf? 
% Sind allen Meilensteinen Aktivit�ten zugeordnet? 
% Gibt es sonstige Inkonsistenzen innerhalb des Zeitplans?
% Welche grundlegenden Chancen und Risiken bestehen f�r Ihr Gesch�ftsvorhaben (bez�glich Technologie, Kundenverhalten, Wettbewerb)?
% In welchen Bereichen Ihres Unternehmens m�ssen Sie beim Eintreten bestimmter Ereignisse in Zukunft Weichen stellen?
Wegen des gesetzlichen Mindestlohns sind wir gezwungen, einen Mindestbetrag an
unsere ungelernten Arbeitskr"afte zu zahlen.
Dies ist jedoch zu Anfang der Produktion ein Problem, da wir nur "uber
begrenztes Kapital verf"ugen.

% Wie wahrscheinlich ist das Eintreten negativer Ereignisse? 
Die M"oglichkeit besteht, dass das Produkt keinen Anklang beim Kunden findet und
somit nur in niedriger St"uckzahl verkauft wird.
Das kann daran liegen, dass bereits Kaffemaschinen und Kaffevollautomaten in
Haushalten und Firmen eingesetzt werden. 
% Wie k�nnen Sie diese negativen Ereignisse eventuell schon im Vorfeld verhindern?
Um einen Fehlschlag des Produkts zu verhindern, werden so fr"uh wie
m"oglich Befragungen in den entsprechenden Zielgruppen durchgef"uhrt.
Durch die dadurch gewonnenen Erkenntnisse kann ein Produkt konzipiert werden,
welches den W"unschen der potentiellen Kunden entspricht.

% Wie k"onnen Sie diese negativen Ereignisse eventuell schon im Vorfeld verhindern?}
Um nach der Entwicklung m"oglichst schnell mit dem Verkauf zu beginnen,
werden wir eine intensive Kundenakquise betreiben.
Die kommerziellen Kunden, wie z.B. Firmen und Beh"orden, k"onnen dann ihren
Kunden und Mitarbeitern unser Produkt zur Verf"ugung stellen, damit deren Aufenthalt bzw.
Arbeit angenehmer wird. 
Mit dieser Strategie verkaufen wir nicht nur unser Produkt, sondern es wird auch
gleichzeitig einer gr"o"seren Masse von m"oglichen Kunden bekannt
gemacht. 
Somit k"onnen wir verst"arkt auf Einzelauftr"age von Privatkunden setzen.

% Wie k�nnen Sie durch Anpassung Ihrer Pl�ne diese Auswirkungen begrenzen (bei Risiken) oder nutzen (bei Chancen)?
Es ist geplant, das Produkt laufend weiter zu entwickeln.
Dies hat den Sinn, eventuell auftretende Fehler zu beheben.
Des Weiteren soll sich dadurch das Produkt den Bed"urfnissen des Markts weiter
anpassen, um letztendlich auch eine Akzeptanz bei den Kunden zu halten.

%Wir setzen 
%-> Auf Marktforschung setzen -> Anpassen an Markt

% Wie werden Sie Ihre Pl�ne im Fall des tats�chlichen Eintretens anpassen?
%\missingText{Wie werden Sie Ihre Pl"ane im Fall des tats"achlichen Eintretens
% anpassen?} 
% Sollte das Produkt widererwarten keinen bzw. nur geringen Anklang auf dem
% Markt finden, so muss versucht werden, Marktpr"asenz aufrecht zu erhalten.

% Sehen Sie alternative Reaktionsm�glichkeiten f�r Ihr eigenes Unternehmen?
% \missingText{Sehen Sie alternative Reaktionsm"oglichkeiten f"ur Ihr eigenes Unternehmen?} 
% -> Produkt einstampfen

% Welche Auswirkungen h�tten die m�glichen Ereignisse, wenn Sie nicht reagieren w�rden?
%\missingText{Welche Auswirkungen h"atten die m"oglichen Ereignisse, wenn Sie nicht reagieren w"urden?}
% -> Pleite

% Welche Auswirkungen hat dies auf Kapitalbedarf und Rendite?
%\missingText{Welche Auswirkungen hat dies auf Kapitalbedarf und Rendite?}

% Inwieweit k�nnte eine gr��ere Kapitalbasis helfen?
%\missingText{Inwieweit k"onnte eine gr"o"sere Kapitalbasis helfen?}

% Wie wird im g�nstigsten und ung�nstigsten Fall Ihre Planung f�r die n�chsten Gesch�ftsjahre aussehen?
%\missingText{Wie wird im g"unstigsten und ung"unstigsten Fall Ihre Planung f"ur die n"achsten Gesch"aftsjahre aussehen?}
