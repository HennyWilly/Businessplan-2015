\section{Management und Schl"usselpositionen}
% Sind alle Mitglieder des Gr�ndungsteams erw�hnt? 
% Sind die Kurzprofile aussagekr�ftig: Ausbildung, Berufserfahrung, Erfolge?
Das Unternehmen wird von Alexander Diener, Hendrik Karwanni und Marco Zidorn gegr"undet.
Diese haben die duale Ausbildung zum Mathematisch Technischen Softwareentwickler mit 
dem Bachelor in "`Scientific Programming"' abgeschlossen.
Bei den Herrn Diener und Zidorn geschah dies in Kooperation mit dem Forschungszentrum 
J"ulich GmbH, bei Herrn Karwanni in Koorperation mit der Enrichment Technology 
Company Limited, die ebenfalls in J"ulich ans"assig ist.
Durch die duale Ausbildung zum Mathematisch Technischen Softwareentwickler
haben die Gr"under bereits einige Erfahrung im Bereich der Projektplanung,
Softwareentwicklung und dem Support von Softwareprodukten sammeln
k"onnen. 

% Ist nachvollziehbar, wer f�r was verantwortlich ist und warum?
Die Herren Diener, Karwanni und Zidorn haben Erfahrungen mit dem Programmieren 
von Apps f"ur das Betriebssystem Android in der Programmiersprache \kursiv{Java} 
und k"onnen somit die App zur Steuerung des Produkts f"ur Android-Ger"ate entwickeln.
Herr Diener besitzt zudem Programmiererfahrung mit der Programmiersprache 
\kursiv{Objective-C} und kann deshalb die App zur Steuerung des Produkts f"ur 
Apple-Ger"ate erstellen.
Die Programmierung der Mikrocontroller zur Steuerung des Kaffeevollautomaten
wird von Herrn Karwanni "ubernommen, da dieser w"ahrend seiner Zeit an einem 
Berufskolleg das Programmieren von Mikrocontrollern mittels der
Programmiersprachen \kursiv{Assembler} und \kursiv{C} erlernt hat.
Fehlendes Know-How in den obigen Bereichen kann sich notfalls durch intensive
Recherchen angeeignet werden.

Da die Schwerpunkte von Herrn Zidorn mehr in Richtung Mathematik als in 
Richtung Informatik gehen, erkl"art sich dieser bereit, die Aufgaben im Bereich
Management und Marketing zu "ubernehmen.
Zudem wird Herr Zidorn f"ur die Finanzierung, den Kontakt zu Kunden
und f"ur den Einkauf verantwortlich sein. Au"serdem fallen die Kundenberatung 
und die Serviceleistungen in Herrn Zidorns Verantwortungsbereich.

Alle Gr"under werden vor den ersten Sondierungsgespr"achen mit eventuellen
Gro"skunden an einem Coaching im Bereich Kommunikation und Verkaufsgespr"ach
teilnehmen. 
Dadurch sind alle Partner auf die Verhandlungen mit den Kunden vorbereitet,
sodass viele Fehler in der Anfangsphase vermieden werden k"onnen.

% Habe ich meine Berater und Netzwerkpartner aufgef�hrt?
% \missingText{Habe ich meine Berater und Netzwerkpartner aufgef"uhrt?}

% Sind alle notwendigen Kompetenzen zur Umsetzung der Gesch�ftsidee vorhanden?
Es werden zus"atzlich noch Kompetenzen zur Fertigung der eigentlichen Maschinen
und Tassen ben"otigt, da keiner der Gr"under Erfahrungen damit hat. 
Um jedoch zeitnah die Anforderungen der Kunden zu befriedigen, muss also die
gesamte Fertigung bzw. Teile davon an eine Ger"atefertigungsfirma, wie
beispielsweise FERCAD Elektronik, abgegeben werden.

% Wie und wann werden fehlende Kompetenzen aufgef�llt?
Sobald die ersten Auslieferungen anstehen, werden kurzfristig studentische 
Hilfskr"afte eingestellt, um die rechtzeitige Auslieferung der
Produkte zu gew"ahrleisten.
Wenn das Gesch"aft die ersten Gewinne erwirtschaftet, sollen zudem
neue Mitarbeiter f"ur Programmierung und Support sowie Marketing und
eventuell Fertigung eingestellt werden.
Dieser Prozess soll flie"send ablaufen, sodass bei zunehmender Anfrage immer
wieder neue Arbeitskr"afte eingestellt werden.
