\section{Realisierungsplanung}
% Sind alle wichtigen Ziele/Meilensteine definiert? 
% Stehen die Meilensteine im richtigen zeitlichen Ablauf? 
% Sind allen Meilensteinen Aktivit�ten zugeordnet? 
% Gibt es sonstige Inkonsistenzen innerhalb des Zeitplans?
% 
% Welche grundlegenden Chancen und Risiken bestehen f�r Ihr Gesch�ftsvorhaben (bez�glich Technologie, Kundenverhalten, Wettbewerb)?
% In welchen Bereichen Ihres Unternehmens m�ssen Sie beim Eintreten bestimmter Ereignisse in Zukunft Weichen stellen?
% Wie wahrscheinlich ist das Eintreten negativer Ereignisse? 
% Wie k�nnen Sie diese negativen Ereignisse eventuell schon im Vorfeld verhindern?
% Wie werden Sie Ihre Pl�ne im Fall des tats�chlichen Eintretens anpassen?
% Sehen Sie alternative Reaktionsm�glichkeiten f�r Ihr eigenes Unternehmen?
% Welche Auswirkungen h�tten die m�glichen Ereignisse, wenn Sie nicht reagieren w�rden?
% Wie k�nnen Sie durch Anpassung Ihrer Pl�ne diese Auswirkungen begrenzen (bei Risiken) oder nutzen (bei Chancen)?
% Welche Auswirkungen hat dies auf Kapitalbedarf und Rendite?
% Inwieweit k�nnte eine gr��ere Kapitalbasis helfen?
% Wie wird im g�nstigsten und ung�nstigsten Fall Ihre Planung f�r die n�chsten Gesch�ftsjahre aussehen?
