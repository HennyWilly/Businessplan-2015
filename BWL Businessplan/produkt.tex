\section{Produkt}
% Welchen Kundennutzen hat Ihr Produkt/Dienstleistung?
Bei unserem Produkt "`\produktname{}"' handelt es sich um einen
personalisierbaren Kaffeevollautomaten\footnote{Ger"at, das vollautomatisch
diverse Kaffeevarianten produzieren kann}, der sich frei vom Kunden konfigurieren und steuern l"asst.
% Worin besteht die Innovation Ihrer Idee?
Das Innovative an unserem Produkt ist die M"oglichkeit, den bevorzugten Kaffee
bzw. die bevorzugte Getr"ankeart pro Nutzer zu speichern.
Dazu stellt der Nutzer seine Tasse an die entsprechende Markierung der Maschine.
Daraufhin liest die Maschine den in der Kaffeetasse integrierten Chip und beginnt 
mit dem F"ullen der Tasse.
Dieser NFC\footnote{Near Field Communication: "Ubertragungsstandard zum kontaktlosen 
Austausch von Daten per Funktechnik "uber kurze Strecken}-Chip kann sowohl in 
bereits mitgelieferten Tassen eingearbeitet, aber auch an den Boden einer
handels"ublichen Tasse befestigt werden.
Per App l"asst sich einstellen, welche Getr"ankeart, deren St"arke und welche 
Menge der Nutzer m"ochte. Zus"atzlich dazu k"onnen eventuelle Beigaben angegeben
werden, wie z.B. Milch.

% Welche Ma�nahmen m�ssen umgesetzt werden, welche Anforderungen m�ssen erf�llt sein, um die Ziele zu erreichen?
% Entwicklung abschlie�en

% In welchem Entwicklungsstadium befindet sich Ihr Produkt/Ihre Dienstleistung?
% in gar keinem

% Welche Entwicklungsschritte planen Sie?
Zu allererst muss ein Prototyp der Maschine von einem Modellbauer erstellt werden.
Damit zeitgleich an der Erstellung der Firmware\footnote{Software, die in
elektronische Ger"ate eingebettet ist} des Kaffeevollautomaten sowie der
Steuerungssoftware in Form einer App gearbeitet werden kann, m"ussen
festgelegte Schnittstellen definiert werden.

% Welche Ressourcen ben�tigen Sie f�r die Produktion?
Zur Herstellung der Kaffeevollautomaten sind Pumpen, Durchlauferhitzer,
Mahlwerke und Br"uhgruppen erforderlich. 
Zudem werden Mikrocontroller zur Steuerung der Komponenten ben"otigt.
Es m"ussen NFC-Leseger"ate sowie die passenden NFC-Chips 
bereitgestellt werden.
Au"serdem ist es notwendig, dass \missingText{WLAN oder Bluetooth}-Adapter
verbaut werden, damit die Smartphone-App mit der \produktname{} kommunizieren kann.

% Welche Versionen Ihres Produkts wird es geben?
Es ist geplant, eine Version des Produkts f"ur kommerzielle Kunden
sowie eine Version f"ur Privatanwender anzubieten, die vom Preis her g"unstiger
ist.
Um diese Preissenkung zu erreichen, werden nicht so
leistungsstarke Komponenten, wie z.B. schw"achere Pumpen, in der
kleineren Privatversion verbaut.

% Haben Sie Folge- und Weiterentwicklungen geplant?}
Es wird laufend am Quellcode der Steuerungsapp sowie der Firmware gearbeitet,
um gemeldete Fehler zu beheben.
Des Weiteren wird das Produkt selbst fortlaufend weiterentwickelt, um auf
neue Anforderungen seitens der Kunden zu reagieren.

% Welche Partnerschaften sind ggf. zur vollen Realisierung des Kundennutzens erforderlich?
\missingText{Welche Partnerschaften sind ggf. zur vollen Realisierung des Kundennutzens erforderlich?}
\missingText{-> Sponsoren???}

% Welche Voraussetzungen sind f�r die Herstellung erforderlich?
% Modell, Firmware, App

% Welche Maschinen werden f�r den Produktionsprozess ben�tigt?
\missingText{Welche Maschinen werden f"ur den Produktionsprozess ben"otigt?}
\missingText{Outsourcing der Produktion: evtl. Herstellung in China}

% Wie sichern Sie die Qualit�t Ihrer Produkte?
Bei Verschlei"s der mitgelieferten Tassen lassen sich sowohl einzelne Chips, als
auch neue Tassen separat bestellen. 
Sollte ein Defekt an der Maschine auftreten, werden wir innerhalb der ersten 2
Jahre nach dem Kauf im Rahmen der gesetzlichen Gew"ahrleistung Reperaturen
durchf"uhren und gegebenenfalls eine neue Maschine bereitstellen.

% Ist Ihr Produkt/Ihre Dienstleistung vom Gesetzgeber zugelassen?
% Ja

% Besitzen Sie Patente oder Lizenzrechte?
Da es momentan noch kein entsprechendes Patent f"ur einen Kaffeevollautomaten in
Kombination mit NFC-Technologie gibt, ist zu erw"agen, ein entsprechendes Patent
einzureichen.
Dadurch wird verhindert, dass ein anderer Hersteller ein "ahnliches Produkt
erstellt und vertreibt, das eine Konkurrenz zu unserem Produkt darstellen kann.

% Welche Patente/Lizenzen sind im Besitz von Mitbewerbern?
% Keine Ahnung

% M�ssen Lizenzen erworben werden und von wem und zu welchen Kosten?
NFC verwendet die ungeregelte Frequenz von 13,56 Mhz, sodass keine
Lizenzen notwendig sind und somit keine zus"atzliche Kosten entstehen.
Des Weiteren f"allt der Kaffeevollautomat unter die Maschinenrichtlinie und muss
somit unter anderem mit der CE-Kennzeichnung versehen werden.
