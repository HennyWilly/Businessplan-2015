\section{Marketing-Mix}

Wir m�chten von unseren Kunden als kundenfreundliches Unternehmen wahrgenommen
werden, da dies auf lange Sicht gesehen die beste Werbung
ist.
Unsere Zielgruppen sind in erster Linie sowohl Arbeitnehmer und Arbeitgeber, als
auch Rentner, die Kaffee trinken.
Laut Statistiken trinken haupts�chlich Erwachsene Kaffee, wodurch sie also zu
der bevorzugten Zielgruppe geh�ren.

Da die Personen unserer Zielgruppe schon morgens aktiv sind und Kaffee trinken,
planen wir die Werbung f�r unser Produkt in der Zeit zwischen 7.00 Uhr und 8.00
Uhr im Radio zu senden.
Dazu wird ein 15-Sekunden-Spot erstellt, der t�glich um diese Zeit
"ubertragen wird.
Daf"ur werden 40.000\euro{} Budget f�r das erste Jahr eingeplant. Je nach Erfolg
wird dieses f�r die kommenden Jahre angehoben oder nicht.
Au�erdem wollen wir eine Anzeige in verschiedenen Zeitungen f�r die erste Woche schalten. 
Daf�r werden 10.000\euro{} eingeplant.

Diese Werbung ist essentiell wichtig, da wir ein neues Unternehmen sind und uns
noch kaum jemand kennt.
Dies ist eine M�glichkeit, auf uns aufmerksam zu machen. 
M"oglich ist auch, die ersten 10 Automaten sowie eine entsprechende Menge
Tassen an gro�e Einrichtungen mit vielen Mitarbeitern, wie z.B. dem
Forschungszentrum J�lich, zu verkaufen. Voraussetzung ist, dass der Automat
m�glichst �ffentlich aufgestellt wird.
Sollten die Mitarbeiter Gefallen an dieser Maschine finden, werden sie sie
evtl. auch privat kaufen wollen.
Wir m�ssen einen gewissen Bekanntheitsgrad und Image erreichen, um konstant
Produkte verkaufen zu k�nnen. Dies ist ebenfalls eine Art von Werbung.
Weiterhin werden wir Mengenrabatte anbieten, um gr��eren Unternehmen einen
Anreiz zu geben, unsere Produkte zu kaufen.

Aufgrund der psychologischen Basis wird der Preis f�r unseren
Standard-Kaffeevollauto\-maten auf 549,99\euro{} und f�r unseren
Extra-Kaffeevollautomaten auf 699,99\euro{} gesetzt.
Es gibt viele Kaffeevollautomaten, die g�nstiger sind. Allerdings besitzt
nur einer davon die Funktionen, die unserer zu bieten hat.
Aufgrund dieser Innovation k�nnen wir den Kaffeevollautomaten etwas teurer als
die anderen Anbieter verkaufen.
Sollte das Produkt wider Erwarten wenig Absatz finden, l�sst sich der Preis noch um bis zu 100\euro{} drosseln, was uns einen gewissen Spielraum gibt. 
Sollte der Absatz hoch sein, werden wir die Preise nach dem ersten Jahr dagegen
anheben.

Der Vertrieb erfolgt in erster Linie direkt �ber unseren Online-Shop. Wir
werden unsere Produkte aber auch bei Amazon und Media Markt anbieten.
Auf unserer Website gibt es zus�tzlich noch ein Video zum Anbringen des Chips an
einer Kaffeetasse.
Telefonnummern und E-Mailadressen f�r Fragen und Support sind dort auch zu
finden, ebenfalls eine Rubrik mit Extraw�nschen.

Dort k�nnen Kunden W�nsche f�r einen personalisierten Kaffeevollautomaten
�u�ern, welche vom Bestimmen der Farbe, dem Muster f�r die Tasse bis zum
Implementieren einer neuen Funktion reichen.
Dies wirkt sich nat�rlich auf den Preis aus, was vertraglich abgesichert
werden muss.
