\section{Marketing-Mix}
BRAUCHEN NOCH EIN LOGO O.O
% Wie und als was wollen Sie vom Kunden wahrgenommen werden?
Als kundenfreundliches Unternehmen. Dem Kunden wird nix aufgeschwatzt, sondern nur �ber unser Produkt (sp�ter vllt Produkte) informiert.
% Welche Zusatzleistungen planen Sie zu Ihrem Produkt?
Man k�nnte eine Funktion einbauen, die das Ger�t automatisch sp�lt und das Wasser direkt ableitet. Dies w�rde sich allerdings im Preis bemerkbar machen w�rd und daher erst was wird, wenn unser Produkt einen gewissen Status erreicht hat. Eine Funktion zum automatischen Wasser nachf�llen ist ebemnfalls eine Idee.
% Welche produktbezogenen Kundenw�nsche k�nnten auf Sie zukommen? Wie reagieren Sie darauf? 
k�nnte: neue Funktionen, Farbe, neuer Chip/Tasse,...
Reaktion: Funktionen lassen sich fast alle denkbaren realisieren, solange der Kunde das bezahlen kann und Arbeitskr�fte zur Verf�gung stehen. Chips und Tassen kann man seperat kaufen und die Farben im Online-Shop selbst bestimmen, ebenfalls eventuelle Muster auf der Tasse.
% 
% Haben Sie einen Preis bzw. eine Preisspanne festgelegt?
sollten wir mal langsam
% Wie sind Sie auf diesen Preis gekommen?
Der Henny der wei� alles!
% F�r welche Preissetzungsstrategie haben Sie sich entschieden? Warum?
.....? :-)
% Wie werden Sie Ihre Preisstrategie im Verlauf der Zeit �ndern?
Preise erh�hen, sobald der Status stimmt.
% Wollen Sie Rabatte gew�hren?
Wenn sich die Konkurrenz einschaltet: JA!
% 
% Welche Vertriebskan�le eignen sich f�r Ihr Produkt?
Meint man hier Internet, Gesch�ft etc? Wenn ja, Internet, sp�ter auch in M�rkten.
% Wie gro� ist die Zahl der potenziellen Kunden, die Sie pro Vertriebskanal erreichen k�nnen?
Zahlen stehen bei Branche&Markt, aber wat f�rn Vertriebskanal???
				% Welche Art des Einkaufens bevorzugen Ihre Kunden?
% Muss Ihr Produkt erkl�rt werden?
Wird ausf�hrlich mit Video auf unserer Website erkl�rt.
% Hat die Preis- und Produktstrategie einen Einfluss auf die Wahl des Vertriebs?
Sagt mir nix, muss ich googeln
% Kann das Produkt l�ngere Zeit gelagert werden?
Ja.
% Wer �bernimmt den Vertrieb? Sie oder eine Spezialfirma?
Erstmal �bernehmen wir den Vertrieb. Wenn wir wachsen schalten wir andere Firmen ein.....w�rd ich sagen, kp ob man das so macht.
% 
% In welchem Ausma� ist Werbung f�r Ihr Produkt wichtig?
Sehr, wir sind Neu im Gesch�ft.
				% Welchen Etat planen Sie f�r Ihre Werbema�nahmen?
% Wie verteilt er sich auf die einzelnen Medien?
Der gro�teil unserer Kunden wird gesch�ftst�tig oder zuhause sein. Werbung �ber Raadio und der Tageszeitung sollte sich daher auszahlen, ebenfalls Werbebanner.
% Wieso wird der Etat so verteilt, wie Sie ihn verteilt haben?
Weil dieser Anteil vermutlich Autof�hrt(Radio) und morgens die Zeitung liest(Tageszeitung).
% Welche Erfolge erhoffen Sie sich von der Kommunikationskampagne?
Bekannt zu werden.
% Haben Sie Medienpartner?
Hab geh�rt Google ist mein Freund....gidf.de