\section{Marketing-Mix}
BRAUCHEN NOCH EIN LOGO O.O
% Wie und als was wollen Sie vom Kunden wahrgenommen werden?
Als kundenfreundliches Unternehmen. Dem Kunden wird nix aufgeschwatzt, sondern
nur "uber unser Produkt (sp"ater vllt Produkte) informiert.
% Welche Zusatzleistungen planen Sie zu Ihrem Produkt?
Man k"onnte eine Funktion einbauen, die das Ger"at automatisch sp"ult und das
Wasser direkt ableitet. Dies w"urde sich allerdings im Preis bemerkbar machen
w"urd und daher erst was wird, wenn unser Produkt einen gewissen Status erreicht
hat. Eine Funktion zum automatischen Wasser nachf"ullen ist ebemnfalls eine
Idee.
% Welche produktbezogenen Kundenw�nsche k�nnten auf Sie zukommen? Wie reagieren Sie darauf? 
k"onnte: neue Funktionen, Farbe, neuer Chip/Tasse,...
Reaktion: Funktionen lassen sich fast alle denkbaren realisieren, solange der
Kunde das bezahlen kann und Arbeitskr"afte zur Verf"ugung stehen. Chips und
Tassen kann man seperat kaufen und die Farben im Online-Shop selbst bestimmen,
ebenfalls eventuelle Muster auf der Tasse.
% 
% Haben Sie einen Preis bzw. eine Preisspanne festgelegt?
sollten wir mal langsam
% Wie sind Sie auf diesen Preis gekommen?
Der Henny der wei"s alles!
% F�r welche Preissetzungsstrategie haben Sie sich entschieden? Warum?
.....? :-)
% Wie werden Sie Ihre Preisstrategie im Verlauf der Zeit �ndern?
Preise erh"ohen, sobald der Status stimmt.
% Wollen Sie Rabatte gew�hren?
Wenn sich die Konkurrenz einschaltet: JA!
% 
% Welche Vertriebskan�le eignen sich f�r Ihr Produkt?
Meint man hier Internet, Gesch"aft etc? Wenn ja, Internet, sp"ater auch in
M"arkten.
% Wie gro� ist die Zahl der potenziellen Kunden, die Sie pro Vertriebskanal erreichen k�nnen?
Zahlen stehen bei Branche \& Markt, aber wat f"urn Vertriebskanal???
				% Welche Art des Einkaufens bevorzugen Ihre Kunden?
% Muss Ihr Produkt erkl�rt werden?
Wird ausf"uhrlich mit Video auf unserer Website erkl"art.
% Hat die Preis- und Produktstrategie einen Einfluss auf die Wahl des Vertriebs?
Sagt mir nix, muss ich googeln
% Kann das Produkt l�ngere Zeit gelagert werden?
Ja.
% Wer �bernimmt den Vertrieb? Sie oder eine Spezialfirma?
Erstmal "ubernehmen wir den Vertrieb. Wenn wir wachsen schalten wir andere
Firmen ein.....w"urd ich sagen, kp ob man das so macht.
% 
% In welchem Ausma� ist Werbung f�r Ihr Produkt wichtig?
Sehr, wir sind Neu im Gesch"aft.
				% Welchen Etat planen Sie f�r Ihre Werbema�nahmen?
% Wie verteilt er sich auf die einzelnen Medien?
Der Gro"steil unserer Kunden wird gesch"aftst"atig oder zu Hause sein. Werbung
"uber Radio und der Tageszeitung sollte sich daher auszahlen, ebenfalls
Werbebanner.
% Wieso wird der Etat so verteilt, wie Sie ihn verteilt haben?
Weil dieser Anteil vermutlich Autof"ahrt(Radio) und morgens die Zeitung
liest(Tageszeitung).
% Welche Erfolge erhoffen Sie sich von der Kommunikationskampagne?
Bekannt zu werden.
% Haben Sie Medienpartner?
Hab geh"ort Google ist mein Freund....gidf.de