\section{Marketing-Mix}

Wir m�chten von unseren Kunden als kundenfreundliches Unternehmen wahrgenommen, da dies auf lange Sicht gesehen die beste Werbung, dank Mundpropaganda, ist.
Unsere Zielgruppe sind in erster Linie sowohl Arbeitnehmer und Arbeitgeber, als auch Rentner, also erwachsene Leute, die Kaffee trinken. 
Laut Statistiken trinken haupts�chlich erwachsene Leute Kaffee, wodurch sie zu der bevorzugten Zielgruppe geh�ren. 
Auch ist unser Produkt nicht mit dem Taschengeld eines Grundsch�lers zu bezahlen, wodurch diese Gruppe von uns nicht weiter beachtet wird.

Da Kaffee meist morgens getrunken wird und die meisten Leute unserer Zielgruppe morgens selber zur Arbeit fahren, planen wir die Werbung f�r unser Produkt um die Zeit zwischen 7.00 Uhr und 8.00 Uhr morgens im Radio laufen lassen. 
Dazu wird ein 15-Sekunden-Spot erstellt, den wir t�glich um diese Zeit laufen lassen. 
Dazu werden \missingText{40.000} Budget f�r das erste Jahr eingeplant, je nach Erfolg wird dieses f�r die kommenden Jahre angehoben oder nicht.
Au�erdem wollen wir eine Anzeige in verschiedenen Zeitungen f�r die erste Woche schalten. 
Daf�r werden \missingText{10.000} eingeplant.
Diese Werbung ist essentiell wichtig, da wir ein neues Unternehmen sind und uns keiner kennt. 
Dies ist unsere M�glichkeit, die Leute auf uns aufmerksam zu machen. 
Es besteht zudem die M�glichkeit die ersten 10 Automaten, sowie einer entsprechenden Menge Tassen, an gro�en Einrichtungen mit vielen Mitarbeitern, wie z.B. dem Forschungszentrum J�lich, zu verkaufen, unter der Voraussetzung, dass der Automat m�glichst �ffentlich steht. 
Sollten die Mitarbeiter gefallen an dieser Maschine entdecken werden sie sie evtl. auch privat kaufen.  
Wir m�ssen einen gewissen Bekanntheitsgrad und Image erreichen, um konstant Produkte verkaufen zu k�nnen und dies ist ebenfalls eine Art von Werbung. 
Weiterhin werden wir Mengenrabatte geben, damit gr��ere Unternehmen eher gewillt sind unsere Produkte zu kaufen.\\
\\
F�r die psychologische Grenze wird der Preis f�r unseren Standart-Kaffeeautomaten auf 549,99\euro{} und der f�r unseren Extra-Kaffeeautomaten auf 699,99\euro{} gesetzt. 
Es gibt zwar viele Kaffee-Vollautomaten, die g�nstiger sind, allerdings besitzt au�er einer davon keine die Funktionen, die unsere hat. 
Aufgrund dieser Innovation k�nnen wir sie etwas teurer als die anderen Anbieter verkaufen. 
Sollte das Produkt wider Erwarten wenig Absatz finden, l�sst sich der Preis noch um bis zu 100\euro{} drosseln, was uns einen gewissen Spielraum gibt. 
Sollte der Absatz stark sein, werden die Preise nach dem ersten Jahr dagegen erh�ht.

Der Vertrieb erfolgt in erster Linie als Direktverk�ufer �ber unseren Online-Shop. Wir werden unsere Produkte aber auch bei Amazon und Media Markt anbieten. 
Auf unserer Website kann man sich unsere Produkte angucken und kaufen. 
Dort gibt es zus�tzlich noch ein Video zum Anbringen des Chips an einer Kaffeetasse, sowie dem Herunterladen und Benutzen der App. 
Telefonnummern und E-Mailadressen f�r Fragen und Support ist dort auch zu finden. 
Ebenfalls eine Rubrik mit Extraw�nschen:

Dort k�nnen Kunden gewisse W�nsche f�r eine personalisierten Kaffee-Vollautomaten �u�ern. 
Das kann von dem Bestimmen der Farbe, ein Muster f�r die Tasse bis zum Implementieren einer neuen Funktion (z.B. mehr als nur 1 gespeichertes Getr�nk auf Chip oder Fernsteuerung, zum Starten des Automaten). 
Dies wirkt sich nat�rlich im Preis aus, wodurch wir im Kontakt mit dem Kunden bleiben und vor dem Anfangen der Arbeit einen Vertrag abschlie�en m�ssen.