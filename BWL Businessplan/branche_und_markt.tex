\section{Branche und Markt}

In Deutschland leben 80,62 Millionen Menschen, von denen ca. 88,3\% Kaffee trinken. 
75\% des Kaffees wird in den eigenen vier W�nden getrunken. 
Damit trinken beinahe 60 Millionen Menschen ihren Kaffee zu Hause. 
Ein drittel davon trinkt ihn t"aglich. Von allen Kaffeetrinkern, die zu Hause
trinken, sind 57,2\% berufst�tig.
Diese Personen geh�ren zu der bevorzugten Zielgruppe unserer Werbung.
Das Interesse am Markt ist also gewaltig.

Gehen wir davon aus, dass mehrere Menschen in einem Haushalt die selbe Kaffeemaschine benutzen, so sollten ca. 20 Millionen Maschinen genutzt werden. 
Wenn wir nur 5\% davon ausmachen, w"aren das 1 Millionen verkaufte Maschinen. 
Da davon auszugehen ist, dass Maschinen auch mal kaputtgehen oder man sich
einfach mal was Neues g"onnen m"ochte sowie der Tatsache, dass die Zahl der
Kaffeetrinker in Deutschland seit 2004 rapide zunimmt, werden im Laufe der Jahre noch viel mehr Kaffeemaschinen im Umlauf sein.
Diese Annahmen basieren auf Statistiken der ICO (Internal Coffee Organization,
2014) und der Zeitschrift Spiegel (2010). Aber auch der "`Tchibo Kaffeereport
2014"' wurde eingesehen.

Der Markt f"ur Kaffeemaschinen ist also auf jeden Fall vorhanden und wird nicht
sinken, sondern weiterhin steigen.
Viele Kaffeemaschinenhersteller (Tchibo, Nespresso,..) haben sich auf diesem Markt bereits einen Namen gemacht und nehmen den Gro�teil des Marktanteils ein. 
Wir als neu gegr�ndetes Unternehmen haben es sehr schwer dort Fu� zu fassen und
Kunden dazu zu bringen, unser Produkt zu kaufen. Das liegt daran, dass wir und
unser Produkt noch zu unbekannt sind.

Um erfolgreich verkaufen zu k�nnen, m�ssen wir also erst einmal bekannt werden.
Dazu werden wir die Radiowerbung zu Zeiten nutzen, an denen meist Kaffee
getrunken wird: am Morgen.
Auch werden wir gr��eren Unternehmen unser Produkt anbieten, damit die
Mitarbeiter, also potenzielle Kunden, unseren Kaffeevollautomaten kennenlernen
k"onnen.
Genaueres zu unserer Werbestrategie steht im Kapitel "`Marketing-Mix"'. 

Wenn die Nachfrage nach unserem Produkt gro� ist, wird die Konkurrenz vermutlich
ein �hnliches entwickeln wollen, um ebenfalls von dieser Idee zu profitieren.
Daher ist zu �berlegen, im Vorhinein ein Patent auf dieses Produkt anzumelden. 
Zus�tzlich wird das Produkt auch stetig anhand von Kundenrezensionen verbessert,
d.h. es werden neue Features hinzugef�gt, was unseren Kaffeevollautomaten auch
in 10 Jahren noch beliebt machen sollte.
Auch ist es m�glich, Extraanfertigungen zu ordern, was ebenfalls im Abschnitt
"`Marketing-Mix"' genauer erl�utert wird.

Auch wenn der Markt f"ur Kaffeemaschinen bereits gedeckt ist, k�nnen wir uns
�ber existierende Marktschranken hinwegsetzen und unseren eigenen Preis diktieren, da unser Produkt �ber innovative Funktionen verf�gt.
Die einzige uns bekannte Kaffeemaschine, die unserem Produkt von der
Funktionalit�t nahe kommt, ist "`Mr. Coffee"' von Belkins. Dabei handelt es sich
um eine Kaffeemaschine mit App-Steuerung.
Mit dieser l�sst sich die Kaffeemaschine von der Ferne aus bedienen. Sie ist sowohl mit iOS als auch mit Android bedienbar. 
Allerdings kann man sich auf dieser Machine nicht, wie bei unserer, sein
Lieblingsgetr�nk speichern.
Dies macht unser Produkt �berlegen, da man sowohl mit als auch ohne
Fernsteuerung letztendlich selbst zur Maschine gehen muss, um sich seinen
Kaffee zu holen.
"`Mr. Coffee"' stellt also keine direkte Konkurrenz f"ur uns dar.