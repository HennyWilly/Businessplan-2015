\section{Branche und Markt}
				% Welche �konomischen Entwicklungen beeinflussen Ihre Branche?
				% Wie beeinflusst der Gesetzgeber Ihre Branche?
% Wie vollzieht sich der Wettbewerb?
Wir machen Werbung in Radio, Tageszeitung, etc.--->man wird auf uns aufmerksam--->wir verkaufen erstmal nur �ber Internet
% Welche Strategien werden verfolgt?
Sollte man unsere Idee kopieren werden neue Versionen rausgebracht, mit zus�tzlichen Funktionen.
% Welche Markteintrittsbarrieren bestehen und auf welche Weise lassen sie sich �berwinden?
Es gibt bereits f�hrende Kaffeemaschinenhersteller, wir m�ssen unsere Produkte die ersten 2 Jahre besonders g�nstig anbieten, um uns einen Namen zu machen. Dann k�nnen die Preise erh�ht und Gewinn gewirtschaftet werdenn 
% 
% Welche wichtigen Mitbewerber bieten vergleichbare Produkte/Dienstleistungen an?
Belkin! Bietet auch eine Kaffeemaschine an, die sich per App bedienen l�sst, aber w�hrend seine nur f�r Android funktioniert wird unsere sowohl f�r Android als auch Apple zu haben sein. Wir werden au�erdem mehr Funktionen einbauen, wie einer Reinigung auf Knopfdruck (noch zu �berpr�fen)
% Wie unterscheidet sich Ihr Angebot von dem ihrer Mitbewerber?
Lies obigen Text, bitch..
					% Wie nachhaltig wird Ihr Wettbewerbsvorteil sein? Warum?
% Wie werden die Mitbewerber auf Ihren Markteintritt reagieren? Wie wollen Sie diese Reaktion beantworten?
Steht auch oben: Neue Funktionen und Farben f�r die Maschine, Rabattaktionen bieten auch ne M�glichkeit.
% 				
% Wer sind Ihre Zielgruppen?
Alle bequemen Kaffeetrinker. Daher meist Personen �ber 18, also keine Kinder, eher Arbeitst�tige, Studierende, Auszubildende, Rentner und M�tter. Allerdings f�r normale Haushalte, wir bieten kein Luxusprodukt zum 10fachen Preis an, dass sich nur die Wohlhabenden holen w�rden. Wir stehen hinter dem kleinen Mann. Allerdings k�nnen Sonderanfertigungen auf Wunsch kreirt werden.
% Welche Kundenbeispiele k�nnen Sie anf�hren?

% Wie gro� ist das Potenzial?
Sehr gro�:
80,62 Millionen Menschen leben in Deutschland, davon trinken 88, 3 Prozent Kaffee, 75 Prozent des Kaffees wird drinnen getrunken. Damit trinken beinahe 60 Millionen Menschen ihren Kaffee zuhause. Ein drittel davon trinkt t�glich.
Gehen wir davon aus, dass mehrere Menschen in einem Haushalt die selbe Kaffeemaschine benutzen, so sollten doch ca. 20 Millionen Maschinen im Umlauf sein/ben�tigt werden. Wenn wir nur 5 Prozent davon ausmachen, w�ren das 1 Millionen verkaufte Maschinen. Da davon auszugehen ist, dass Maschinen auch mal kaputtgehen, oder man sich was neues g�nnen m�chte, und der Tatsache, dass die Zahl der Kaffeetrinker in Deutschland seit 2004 rapide zunimmt, werden im Laufe der Jahre noch viel mehr verkauft.
Der Bedarf ist also auf jeden Fall vorhanden.
% Auf welchen Annahmen basieren Ihre Sch�tzungen?
Statistiken der vorherigen Jahre, unter anderem durchgenommen von ICO, der "Internal Coffee Organizaton".
% Unter welchen Annahmen lassen sich Ihre Sch�tzungen verallgemeinern?
...das stimmt schon so.
% Was sind die kaufentscheidenden Faktoren? 
Wir bieten was neues an, was das Leben der Leute vereinfacht, und liegen mit dem Preis dennoch bei dem der anderen Anbieter. Dass die Kaffeemaschine Pink-Gelb ist ist ebenfalls ein entscheidender Faktor.
				% Welche Rolle spielen Service, Beratung und Wartung?
				% Gibt es Pilotkunden?
