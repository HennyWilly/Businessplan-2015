\section{Gesch"aftssystem und Organisation}
% Welche Rechtsform soll Ihr Unternehmen haben? 
% Ist Ihre Wertsch"opfungskette in sich logisch geschlossen?
% In welchen Wertsch"opfungsabschnitten verdienen Sie Geld?
% Wie l"auft der Kundenzahlungsprozess ab?
% Wie viele F"uhrungspositionen hat Ihr Unternehmen?
% Welche Firmen- und Mitarbeiterkultur streben Sie an?
\def \name {CaffeMac}
\def \startkapital {25.000\euro}

\subsection{Rechtsform}
Die \name UG wird sp"ater in ein Handelsregister eingetragen. Diese Rechtsform wurde gew"ahlt um die Risiken
zu begrenzen und bei Kunden und Zulieferern als ?normales? Unternehmen wahrgenommen zu werden.

Alexander Diener, Hendrik Karwanni und Marco Zidorn "ubernehmen jeweils ein Drittel des Stammkapitals der Gesellschaft.

Das Stammkapital von \startkapital\ wird bar eingebracht. Es wird zum Teil f"ur die Prototypenrealisierung und f"ur erste Investitionsg"uter ben"otigt, die dem Unternehmen als Anlageverm"ogen zur Verf"ugung stehen. Der nicht sofort ben"otigte Teil ist Liquidit"atsreserve.

\subsection{Wertsch"opfungskette}
Am Beginn steht die Bestellung der Ger"ate bei einer Ger"atefertigungsfirma, wie beispielsweise FERCAD Elektronik, die einen guten Ruf am Markt hat. Anschlie"send folgt ein Verkauf und Versand der Produktion innerhalb Deutschland.

\subsection{Kundenzahlungsprozess}
Der Zahlungsprozess wird in Form einer "Uberweisung auf Rechnung erfolgen.

\subsection{Firmen- und Mitarbeiterkultur}
Es wird ein Leiter und zwei Stellvertreter geben.
Die Leitung der Firma "ubernimmt Hendrik Karwanni und auf zwei stellvertretenden Positionen kommen Marco Zidorn und Alexander Diener.
Weitere Mitarbeiter sind nicht n"otig, da alles andere, wie Erstellung oder Zustellung der Produktion durch passende Unternehmen/Firmen erledigt wird.