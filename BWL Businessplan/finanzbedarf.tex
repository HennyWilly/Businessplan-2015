\section{Kapitalbedarf und Finanzierungsplanung}
% Geht aus Ihrem Businessplan hervor, wie viel Geld Sie ben�tigen?
% Brauchen Sie eine einmalige Finanzierungsspritze oder mehrere Finanzierungsrunden?
% Wen wollen Sie ansprechen? Fremdkapitalgeber oder Eigenkapitalgeber?
% Welche Sicherheiten k�nnen Sie ggf. bieten?
% Haben Sie sich mehrere Angebote eingeholt?

\begin{tabular}{ll}
	\hline 
	 Eigenmittel Gr"under 		& 25.000\euro \\ 
	 Mittel einer Privatperson 	& 75.000\euro \\ 
	\hline  
	 Kapitalbedarf 				& 100.000\euro \\ 
	\hline 
\end{tabular} 
\\[15pt]
Der Gesamtkapitalbedarf von 100.000\euro\ wird zu 100\% aus Eigenmitteln gedeckt. 
Diese setzen sich aus dem Kapital der Gr"under und einem Kredit einer Privatperson zusammen.
\par~\\
{\large\textbf{Zinsplanung}}\\
Die in das Unternehmen integrierten Kreditmittel einer Privatperson werden mit 5,0\%
verzinst. 

\subsection{Drei-Jahres-Planung}
Da sollen alle m"ogliche Verluste und Gewinne erkl"art werden...
% Welchen Personalbedarf und welche Personalkosten erwarten Sie in den einzelnen Bereichen Ihres Unternehmens in den n�chsten f�nf Gesch�ftsjahren?
% Welche L�hne bzw. Geh�lter wollen Sie zahlen?
% Gibt es sonstige Zusatzleistungen f�r Ihre Mitarbeiter, z. B. 13. Gehalt oder Bonuszahlungen?
% 
% Ist die Gewinn- und Verlustrechnung f�r das erste Jahr auf Monatsbasis in der F�nf-Jahres-Planung dargestellt?
% Wird die Gewinn- und Verlustrechnung im zweiten Jahr nach jedem Quartal erhoben? 
% 
% Haben Sie f�r das erste Gesch�ftsjahr Auszahlungen und Einzahlungen monatsweise aufgeschl�sselt?
% Sind Einzahlungen und Auszahlungen f�r das zweite und dritte Gesch�ftsjahr in der F�nf-Jahres-Planung nach Quartalen aufgeschl�sselt?
% Sind ab dem vierten Gesch�ftsjahr die Einzahlungen und Auszahlungen per Anno aufgeschl�sselt?   
% Ab wann rechnen Sie mit einem Einzahlungs�berschuss (Summe aller Einzahlungen gr��er Summe aller Auszahlungen)?
