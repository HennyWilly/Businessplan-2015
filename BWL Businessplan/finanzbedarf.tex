\section{Finanzbedarf}
% Geht aus Ihrem Businessplan hervor, wie viel Geld Sie ben�tigen?
% Brauchen Sie eine einmalige Finanzierungsspritze oder mehrere Finanzierungsrunden?
% Wen wollen Sie ansprechen? Fremdkapitalgeber oder Eigenkapitalgeber?
% Welche Sicherheiten k�nnen Sie ggf. bieten?
% Haben Sie sich mehrere Angebote eingeholt?

\begin{tabular}{ll}
	\hline 
	 Eigenmittel Gr"under 		& 45.000\euro{} \\ 
	 Mittel einer Privatperson 	& 15.000\euro{} \\ 
	\hline  
	 Kapitalbedarf 				& 60.000\euro{} \\ 
	\hline 
\end{tabular} 
\\[15pt]
Der Gesamtkapitalbedarf von 60.000\euro{} wird zu 100\% aus Eigenmitteln gedeckt. 
Diese setzen sich aus dem Kapital der Gr"under und einem Kredit einer Privatperson zusammen.\\
Die Kreditmittel der Privatperson werden zur Sicherheit des Einstiegs mit 10,0\% Zinssatz f�r 3 Jahren geliehen. \\
Im ersten Halbjahr werden 3.000\euro{} notwendig sein, um jedem Partner die
Grundausstattung f�r die Weiterentwicklungsarbeiten zur Verf�gung zu stellen. Diese
Ausgaben beschr�nken sich im Wesentlichen auf Computer-Hardware und den
B�robedarf.\\
Im zweiten Halbjahr werden nochmals etwa 1.500\euro{} notwendig sein um die
IT-Infrastruktur im angemieteten Firmenb�ro herzustellen, um die Entwicklung
unserer Ger�te zu gew�hrleisten und das B�ro zu m�blieren. Zudem wollen
wir im zweiten Halbjahr des ersten Gesch�ftsjahres einen Firmenwagen mieten, der monatlich mit 100
Euro veranschlagt ist.\\
Wie aus der Liquidit�tsrechnung erkennbar ist, werden diese Ausgaben zun�chst aus der Eigenkapitaleinlage der Gesch�ftsf�hrer gedeckt. Der Kredit und
der Rest des Eigenkapitals werden zur Beschaffung der Ger�te verwendet. Das f�r das zweite und das dritte Gesch�ftsjahr ben�tigte Kapital soll aus den Einnahmen aufgebracht
werden. Da unser Unternehmen Ende des ersten Jahres die ersten Gewinne erzielt, sollte das Kapital ausreichen. Falls diese Planung nicht entsprechend erf�llt werden kann, werden bei den Privatperson zus�tzliche Summen geliehen. Diese Option erh�ht
zwar die zu zahlenden Zinsen, kann aber in schwierigen Situationen die Zahlungsf�higkeit unseres Unternehmens garantieren.
