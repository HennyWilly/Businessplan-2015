\section{Executive Summary}
% Was genau ist Ihre Gesch�ftsidee? 
% Haben Sie Ihre Gesch�ftsidee in leicht verst�ndlicher Weise formuliert?
Unser Gesch"aft ist der Vertrieb von neuartigen
Kaffeevollautomaten.
Dies sind erweiterte Kaffeemaschinen, die neben einem Heizk"orper zu Erhitzen
des Wassers zus"atzlich auch ein Mahlwerk, einen Durchlauferhitzer, eine
Pumpe und eine Br"uhgrube besitzen. 
Dadurch k"onnen per Knopfdruck auch komplexere Getr"anke, wie z.B. Cappuccino,
Latte Macchiato oder auch Kakao zubereitet werden.

% Was ist das besondere an Ihrer Idee im Vergleich zu bereits existierenden Gesch�ftsideen?
% Inwieweit besitzt Sie Alleinstellungsqualit�t?
Das Besondere an unserem Produkt ist die M"oglichkeit, mittels eines
NFC\footnote{Near Field Communication: "Ubertragungsstandard zum kontaktlosen
Austausch von Daten per Funktechnik "uber kurze Strecken}-Chips, der in den
mitgelieferten Tassen verbaut ist und eines entsprechenden Leseger"ats in der
Maschine, zu erkennen, welches Getr"ank zubereitet werden soll.
Dadurch muss der Benutzer selbst keine Eingabe an der Maschine t"atigen.
Dies wird mittels einer Smartphone-App realisiert, die es erlaubt, auf einem
bestimmten Chip ein Getr"ank und eventuelle Beigaben zu konfigurieren.

% Worin besteht der eigentliche Kundennutzen Ihrer Gesch�ftsidee?
Wir bieten dem Kunden somit die M"oglichkeit, vollautomatisch sein
favorisiertes Hei"sgetr"ank zuzubereiten, nur dadurch, dass er seine spezielle
Tasse an der entsprechenden Stelle der Maschine platziert.
% Welchen Zielmarkt sprechen Sie an?
Zu den Kunden z"ahlen sowohl Privatpersonen, als auch gewerbliche Kunden, wie
z.B. Firmen und Beh"orden.

% Wie soll Ihr Unternehmen Geld verdienen?
Der Gewinn des Unternehemens setzt sich gr"o"stenteils aus dem Verkauf der
Kaffeevollautomaten zusammen. 
Zus"atzlich dazu werden auch noch einzelne Tassen mit NFC-Chips verkauft, die
besch"adigte Tassen ersetzen.
Eine weitere Einnahmequelle ist der Verkauf von einzelnen Chips, die an den
Lieblingstassen der Kunden angebracht werden k"onnen, um damit auch den
maximalen Komfort zu erhalten.

Hendrik Karwanni ist dabei der Manager, da er gut im organisatorischen Bereich ist, Marco Zidorn ist aufgrund seiner Ideenvielfalt Marketingleiter und Alexander Diener ist unser Leiter f�r Finanzen, da er gut mit Zahlen jonglieren kann.

In den ersten f�nf Jahren wollen wir einen gewissen Bekanntheitsgrad erreichen und mindestens 5 \% des Marktanteils ausmachen.

Im ersten Jahr werden wir sehr wahrscheinlich noch im Minus sein, das sollte sich aber bis Mitte des zweiten Jahres �ndern. 